\section{Determinism} \label{sc:determinism}

\subsection{Finite automata for determined/undetermined systems}

Determinism is an important part of real time computing providing much needed predictability to the execution of tasks. When a task requires a predictable amount of computing resources, a more precise schedule can be made improving the overall performance of the system. One method for making a job deterministic is by using finite state automata. This chapter will detail deterministic and timed finite automata for exploring determinism and non-deterministic finite automata for a short discussion on non-determinism and their relationship with real-time computing. \par

\subsection{Deterministic finite automata (DFA)}
A variant of finite automata where each state is allowed to have up to one exit transition from each state for a given member of the alphabet. This makes the execution of the automata very predictable. A DFA is considered complete if there is an exit transition for each member of the valid alphabet. \par

\subsection{Non-deterministic finite automata (NFA)}
A variant of finite automata where each state is allowed to have more than one exit transition for a given member of the alphabet. The automaton will then process each token from the alphabet until it reaches an accepting state. Any NFA can be converted into a DFA using an algorithm known as powerset construction. \par

\subsection{Timed finite automata (TFA)}

Compare deterministic and non-deterministic automata
Talk about timed automata and how they comply with deadlines

