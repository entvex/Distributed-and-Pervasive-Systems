\section{Scheduling} \label{sc:scheduling}

- Introduction
Arrival time, release time, deadlines
Real-time computing
What are jobs? Tasks?
Have the basic definition written down (at least the Task/J one)


In scheduling the aim is to meet all hard deadlines and handle soft deadlines of jobs in the best possible manner and avoid deadlocks while doing so. The arrival time of a job is the moment in time it arrives at a processor, while the
release time of a job is the moment in time it becomes available for execution. A task can have different job types and such as perodic, sporadic and Aperiodic. A task is a set of jobs known at the start of the system or triggered by a external event.
A periodic task is defined by three parameters:
%TODO: Tror du der er en måde at skubbe dem mere sammen, så der er mere plads?
\begin{itemize}
	\itemsep0em
	\item The release time r of the first periodic job.
	\item The period p, which is a periodic time interval, at the start of which a periodic job is released.
	\item The execution time e of each periodic job.
\end{itemize}
and is written as (\textit{r},\textit{p},\textit{e})

\subsection{Rate-monotonic scheduler}
The rate-monotonic scheduling strategy gives a higher priority, to periodic jobs with a shorter period. Because of the static nature of the scheduler, it is easy to compute and predict.\footnote{\cite{Fokkink1965} p.183}

%Example goes here
\subsection{Earliest deadline first scheduler}
%noget af dette er taget fra bogen, men det er menga svlrt at koge mere ned synes jeg.
This scheduling strategy will give a higher priority to jobs if its deadline is
earlier. In case of preemptive jobs and no competition for resources, this scheduler
is optimal, in the sense that if utilization at a processor does not exceed one, then
periodic jobs will be scheduled in such a way that no deadlines are missed.\footnote{\cite{Fokkink1965} p.184}

%Example goes here

\subsection{Least-slacktime-first scheduler}
This scheduling strategy gives higher priority to jobs wit less slack (Idle time). This scheduler is a good choice if utilization at a processor does not exceed one. In this case the periodic jobs will be scheduled in such a way that no deadlines are missed.\footnote{\cite{Fokkink1965} p.184}

%Example goes here
\subsection{Resource control}
Write about resource control/deadlock
Priority inheirtance
Priority ceiling
Find Søren Hansen (grandmaster) drawings/references