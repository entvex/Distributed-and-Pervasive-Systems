\chapter{Consistency} \label{ch:consistency}

Consistency is a condition where all nodes in a system agree on the current state of the system and see the same data at the same time. In a centralized system, this challenge often becomes more manageable, as decisions that change the state of the system are stored in shared memory between processes. In a distributed system with multiple nodes, each processor has its own private memory and the ability to change the state of the system, so state information must be passed from one node to the next. This can create consistency problems, when nodes become out of sync or somehow do not reflect the current state of the system due to server failures, bandwidth bottlenecks or network topology changes. In this chapter we look at the CAP and PACELC theorems for consistency and replication, and discuss various consistency models used in distributed systems.

\section{The CAP theorem}

In-depth about CAP.

\section{PACELC}
The PACELC theorem is a addition to the CAP. In CAP, one must choose between high availability and consistency. This leaves no option for mission-critical but to sacrifice consistency because high availability is a must. This logic is not right because the P in CAP is a mix of \textit{partition tolerance} and a \textit{actual network partition}. Therefor it is wrong to assume the systems that reduce consistency in the absence of any partitions are doing it because of CAP. The PACELC theorem handles the cases when a network partition have or not happened.

The theorem states If there is a partition \textbf{P}, how does the system trade off availability \textbf{A} and consistency \textbf{C}. Else, when the system is running normally in
the absence of partitions, how does the system trade off latency \textbf{L}
and consistency \textbf{C}? \textbf{( REF TIL Fischer DIPS S08 Consistency Slides.pdf side 13 ).}

As it can be seen in table \ref{tab:PACELC} \textbf{Cassandra} is a PA/EL system. So if a partition happens, it will give up consistency for availability, and during normal operation they give up consistency for lower latency. \textbf{BigTable}
Is a ACID system and therefor PC/EC. It will refuse to give up consistency and pay the price in terms availability and latency costs to achieve it.\textbf{MongoDB} can be classified as a PA/EC system. In the default configuration, the system guarantees reads and writes to be consistent. When MongoDB faces a  \textbf{( REF TIL 2012 Abadi ConsistencyTradeoffsInModernDistributedDatabaseSystemDesign.pdf side 42 ).}
%TODO Denne fucking table vil ikke stå under texten!?!?!
When a system have a network partition happen to it there is a few options for replication the data.
\begin{enumerate}
	\item Data updates sent to all replicas at the same time.
	\begin{enumerate}
		\item Using a preprocessing layer gives us consistency but increased latency.
		\item Not using a preprocessing layer will decrease latency but can only offer eventual consistency.
	\end{enumerate}
	\item Data updates sent to an agreed-upon location first
	\begin{enumerate}
		\item Synchronous the master node waits until updates made it to the replicas. Therefor gaining consistency but pays latency.
		\item Asynchronous treats the update as if it were completed before being sent to a replica.
	\end{enumerate}
	\item Data updates sent to an arbitrary location first.
		\begin{enumerate}
		\item Synchronous then the latency problems of (2)(a) are present.
		\item Asynchronous consistency problems same as (1) and (2)(b) are present.
	\end{enumerate}
\end{enumerate}

\begin{table}
	\centering
\begin{tabular}{|c|c|c|c|c|}
	\hline 
	System & A & C & L & C \\ 
	\hline 
	Cassandra & X &  & X &  \\ 
	\hline 
	BigTable &  & X &  & X \\ 
	\hline 
	MongoDB &  & X & X &  \\ 
	\hline 
\end{tabular}
  \caption{This is my one big table} \label{tab:PACELC}
\end{table}

\section{Consistency models}

Consistency models are used in distributed systems like shared memory storage, filesystems and databases to amend the issues that distributed write and read access could have to such a system. A consistency model define a given set of rules a operation must follow when accessing the data and can be seen as a contract between a running process (the programmer) and the system. If the process comply with this contract, then memory will be consistent and the output of reading, writing, or updating the memory will be predictable. For example in relational database systems, transactions abide by the ACID\footnote{ACID (Atomicity, Consistency, Isolation, Durability) is a set of properties of database transactions intended to guarantee validity even in the event of errors, power failures, etc.} principle, where consistency ensures that any transition will bring the database from one valid state to the next and that data should follow a set of rules. Assume a relational database in a cluster that contain row $R_x$ and that this row is replicated to nodes $Node_a$ and $Node_b$. A client $C_i$ then writes $R_x$ on $Node_a$. Subsequently another client $C_j$ reads $R_i$ from $Node_b$. A consistency model has to determine if $C_j$ sees the write from $C_i$ or not.

Consistency models can be classified as strong or weak. A strong consistency model guarantees that the order and visibility of updates are equivalent to that of a centralized (non-replicated) system with only one process, that is updates are immediately replicated to all nodes. A weak consistency model does not make this guarantee. We will now look at both strong and weak models and consider them for a distributed system:

\textbf{Linearizability}: A strong consistency model that ensures that all operations appear to have run atomically (in isolation, independent from concurrent processes) in an order that is consistent with the global real-time ordering of operations. If a distributed database follows this model, than it would appear centralized to its clients. No matter how clients execute operations, the result would be the same as if they we executed in some sequential order and these operations appear in their order of executing.

\textbf{Sequential consistency}: A strong consistency model that mimics the linearizability one, but does not require that two independent operations from different clients respect the real-time ordering of operations in a system. With this model, operations can be re-ordered as long as each node observe the ordering consistently. A benefit of the two strong consistency models is that a client or programmer can replace a centralized system with a distributed one without facing data consistency issues.

\textbf{Client-centric consistency:} 

\textbf{Casual consistency}: 

\textbf{Eventual consistency}: A 
