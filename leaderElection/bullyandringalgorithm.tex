\section{The bully and ring algorithm}

The algorithm was proposed by Garcia-Molina in 1982 and simply put elects the node with the highest id out of \textit{N} nodes with unique ids, hence the name. The current leader would be the one with the highest id at time \textit{t} in a star network cluster. When any node $N_i$ notices that the leader is not responding, it starts an election as follows:

\noindent Node $n_i$ sends a ELECTION message to all nodes with higher ids, that is $n_i+1, n_i+2 ... n-1$ nodes. If it does not hear from any, node $n_i$ wins the election. If any node with a higher id responds, it will take of the election process and $n_i$ will not contribute further. During this process our node $n_i$ can receive a ELECTION message from a node with a smaller id. It will acknowledge the message and begin a new election, if one is not already taking place. This continues until basically \textit{N-1} has given up, leaving the last node as the new leader or coordinator. The node will put a message to all other nodes announcing its leadership. If a node with a higher id is added to the cluster, it will take the reins the conduct a new election, which it will win.

\noindent The benefits of this algorithm is clearly is intuitiveness, simplicity and its fairly low amount of traffic generated to elect a leader, but may lack a more sophistic approach to it if the node with the highest id is unreliable or unstable. One could imagine the leader quickly becoming overloaded with leader obligations, such as replicating a data log to \textit{N} nodes and serving clients, which would simply trigger a whole new election.

\noindent The ring algorithm works similarly to the bully, but has nodes in a logical ring topology rather than a star. Each node knowns its successor. If a node does not receive a sign of life from the leader, it starts an election by sending an ELECTION message containing its id to its successor. The successor adds its id to the message and forwards it. This continues for \textit{N-1} nodes until the messages gets back to the initiating node, which then sends out a COORDINATOR messages informing nodes of who the leader is (the one with the highest id) and which nodes are part of the ring.