\section{Logical Clock}\label{sc:logicalClock}

Good for managing ordering and scheduling of events in a system without needing to exist within the domain of real time. Time is defined in terms different from \textit{t(x)}

\subsection{The need for logical clocks}

\subsection{The happened-before relation}

Define the happened-before relation. I guess most can be found in the article "Time, Clocks, and the Ordering of Events in a Distributed System" or in the slides.

\subsection{Lamport time-stamps}

Definitions from Lamport
a->b
Concurrency

- Invented by Leslie Lamport in 1978.
- It is an algorithm to determine the ordering of events in a system.
- Simply an local counter that each process increments before an event and attaches when it sends a message. The receiving party compares the received value with its own and whichever value is the greatest is then assigned to the counter of the received process.

Pros/Cons

Lamport clock - The one he talked about in the lecture.
A sample could be prepared from the BlinkToRadio example in sensor networks.
OR
To evaluate this is in practical setting, we define a simple Python application using MPI (Message Passing Interface) that demonstrates Lamport time-stamps in action.
-- SHOW code samples from python code.
OR
Heartbeat clock - A clock which is ticked forward based on a 'heartbeat' that all parts of the system have access too.
Can't find any sources for this quickly. It may need a bit of extra 