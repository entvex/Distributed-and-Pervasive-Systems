\section{The basics of fault tolerance}

A major difference between centralized and distributed systems is that we can experience partial failures in a distributed one. Suddenly, nodes can malfunction or become inaccessible due to network partitions. This can lead to decreased usability for clients. Therefore a strategy needs to be in place that can handle these kinds of failures. Without the seriousness of the failure would be proportional to the decline in quality offered to clients. Before diving into what types of failures we may witness, it is important to understand what values we seek in a distributed system and how they differ from each-other.

\noindent \textbf{Availability}: Availability is the quality of being able to be used right away. It is defined as a probability that the system will be ready when users need it to be. For high-availability or mission-critical systems, we usually see an uptime around five nines, which means the system has a probability of 99.999\% to be ready when users call.

\noindent \textbf{Reliability}: Reliability is the quality of being trustworthy and be able to produce consistency in measurements, i.e. the system will run without failures. It is defined as a time interval instead of a instance in time. If a system has high availability but still crashes randomly at times, it is unreliable. If a system never crashes, but is often shut down due to maintenance at a specific time, it is reliable.

\noindent \textbf{Safety}: To archive safety in a system, nothing fatal can happen should the system experience a failure. We often see this in mission-critical systems, where even a brief outage can have severe consequences. This could be a space rocket or a power plant facility. In less crucial systems, safety may not play such a huge role.

\noindent \textbf{Maintainability}: Maintainability tells us how easy a failed system can be examined, repaired and have worn-out components replaced by newer ones. We often see a correlation between high maintainability and high availability. 

\noindent \textbf{Atomicity}: Finally, atomicity is the quality of a operation or transaction that is indivisible and irreducible. Either it completes fully or nothing happens at all. This correlates to safety and can be seen especially in banking systems, where a money transaction between two accounts often involves multiple system and must be reversible should something interfere.

\noindent A system is said to fail when it cannot meet its promises or is unable to perform its required function. Failures are sometimes caused by errors, but they do not have to lead to failures, however. The cause of an error is a fault, that is an incorrect step, process or definition in a software application which causes the system to perform in an unintended manner. An technical fault in a branch of code on a node could lead to it experiencing an error, which could make it crash during execution. If other parts of the system depend on this node, the error could lead to a system failures. To classify these we use the following terms:

\noindent \textbf{Crash failures} occur when a node suddenly halts, but was working fine up until that point. A physical action must be taken to bring the node back up. \textbf{Omission failures} occur when a node does not respond to a incoming request. The request can malformed or perhaps the node fail to generate a response. \textbf{Timing failures} happen when a node does not respond within a time limit, maybe due to network delay or excessive load. \textbf{Response failures} is when a node respond, but its response is incorrect or is out of step with the current state of the system. \textbf{Arbitrary failures} are random and can occur at any time. These can be hard to narrow-down and correct.

\noindent To gain fault tolerance in a distributed system, we should seek to mask the occurrence of failures from other processes and have a back-up plan ready (redundant nodes) in case they should happen. We should also implement self-stabilization, so that the system will converge towards a correct state, no matter the initial one. Consensus is the key to archive this.